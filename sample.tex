% Compile this file with xelatex. You have to have the font DejaVu Sans installed on your system!

\documentclass{article}
\usepackage[english]{babel}
\usepackage{geometry}
\usepackage{
alttex,
blindtext,
dtklogos,
fancyhdr,
multicol,
lmodern,
shortvrb,
xltxtra
}

\MakeShortVerb|

\newenvironment{sample}[1]{
  \subsection*{#1}
  \hrule
  \label{#1}
  \begin{minipage}{\textwidth}
  \begin{multicols}{3}}
{\end{multicols}
  \end{minipage}\vspace{2cm}}

\fancyhead{\normalfont\qquad alttex\hfill normal \hfill result\qquad}
\pagestyle{fancy}

\setmonofont[Scale=0.8]{DejaVu Sans} % the standard fonts don’t have the nice symbols!

\title{A short sample document for the |alttex| package}
\author{Arno L. Trautmann\thanks{arno.trautmann@gmx.de}}
\date{\today}

\begin{document}
\maketitle
This document shows applications of the possibilities the |alttex| packages offers for writing code in \XeLaTeX.

There are three columns: The left one shows the input code using |alttex|, the middle one shows the corresponding ”normal“ input, and on the right-hand side there is the output of that code. The output is written in the code on the left-hand side, so I am not cheating here…\vspace{2cm}

\sample{noescape}
\noescape
\begin{verbatim}
This is a short text including
some characters like an @,
a §, 100% unescaped, even {}
and ~ or a &
without a backslash.
\end{verbaatim}
\vfill
\oldescape
\begin{verbatim}
This is a short text including
some characters like an @,
a §, 100\% unescaped, even \{\}
and \textasciitilde\ or a
\& without a backslash.
\end{verbatim}
\vfill
\noescape
This is a short text including
some characters like an @,
a §, 100% unescaped, even {}
and ~ or a &
without a backslash.
\oldescape
\endsample
\makealtmath

\sample{general math}
|\[∫_-∞ ^∞ ∀\]|

\ \vfill
|\[\int_{-\infty}^{\infty}\]|
\ \vfill
\[∫_-∞ \ \]% ^∞ ∀\]
\endsample

\sample{itemize}
\begin{verbatim}
A short sample text

• first item
• second item

  ‣ deeper item
  ‣ second deeper item
  \end{itemize}
\end{itemize}
\end{verbatim}
\ \vfill
\begin{verbatim}
A short sample text
\begin{itemize}
\item first item
\item second item
\begin{itemize}
  \item deeper item
  \item second deeper item
  \end{itemize}
\end{itemize}
\end{verbatim}
\ \vfill
A short sample text

• first item
• second item

  ‣ deeper item
  ‣ second deeper item
  \‣
\•
\endsample

\sample{enumerate}
\begin{verbatim}
A short sample text

¹ first item
¹ second item

  ² deeper item
  ² second deeper item
  \end{enumerate}
\end{enumerate}
\end{verbatim}
\vfill
\begin{verbatim}
A short sample text
\begin{enumerate}
\item first item
\item second item
\begin{enumerate}
  \item deeper item
  \item second deeper item
  \end{enumerate}
\end{enumerate}
\end{verbatim}
\vfill
A short sample text

¹ first item
¹ second item

  ² deeper item
  ² second deeper item
  \end{enumerate}
\end{enumerate}
\endsample

\sample{huge display math}
\begin{verbatim}
\begin{hugedisplaymath}
E = mc^2
\end{hugedisplaymath}
\[E = mc^2\]
\end{verbatim}
\vfill
\begin{verbatim}
\Huge
\[ E = mc^2\]
\normalsize
\[E = mc^2\]
\end{verbatim}
\vfill
\begin{hugedisplaymath}
E = mc^2
\end{hugedisplaymath}
\[E = mc^2\]




\endsample
\end{document}
