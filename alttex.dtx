% \iffalse
%% File: alttex.dtx by Arno Trautmann, mail: arno dot trautmann at gmx dot de
%<*internal>
\iffalse
%</internal>
%<*readme>
This is the readme file.
%</readme>
%<*internal>
\fi
\begingroup
%</internal>
%<*batchfile>
\input docstrip.tex
\keepsilent
\preamble
EXPERIMENTAL CODE

Do not distribute this file without also distributing the
source files specified above.

Do not distribute a modified version of this file under the same name.

\endpreamble
\postamble
Copyright 2008–2009 Arno Trautmann <arno.trautmann@gmx.de>

Distributed under the LaTeX Project Public License,
verson 1.3c or higher (your choice). The latest version of
this license is at: http://www.latex-project.org/lppl.txt

This work is "author-maintained" by Arno Trautmann

This work conists of this file alttex3.dtx
         and the derived files alttex3.sty
                               alttex3.pdf
                               sample.tex
                           and sample.pdf.

\endpostamble
\askforoverwritefalse

\generate{\file{alttex3.sty}{\from{alttex3.dtx}{alttex}}}
%</batchfile>
%<batchfile>\endbatchfile
%<*internal>
\generate{\file{alttex3.ins}{\from{alttex3.dtx}{batchfile}}}
\nopostamble
\generate{\file{sample.tex}{\from{alttex3.dtx}{sample}}}
\nopreamble
\generate{\file{README}{\from{alttex3.dtx}{readme}}}
\endgroup
%</internal>
%
%<*driver>
\documentclass[a4paper]{ltxdoc}
\usepackage{ifxetex,ifluatex}
\usepackage[english]{polyglossia}
\usepackage{
  expl3,
  todonotes,
  hyperref,
  lmodern,
  xltxtra
}

\hypersetup{%
  pdfborder= 0 0 0,
  colorlinks=true,
  linkcolor= blue,
  pdftitle=altTeX
}

\def\filedate{2010/01/09}
\def\fileversion{0.c}

\title{The \textsf{alttex} package}
\author{Arno L. Trautmann\thanks{arno.trautmann@gmx.de}}
\date{Version 0.c \filedate}

\EnableCrossrefs
\CodelineIndex
\RecordChanges

\usepackage{ifthen}

%\OnlyDescription
\begin{document}
\maketitle
% The following definition has to be made at this strange position for some reasons. It checks whether the font DejaVu Sans is installed on your system or not. If so, it will be used to display some characters the lmodern font does not have. If the font is not present, you will not see the symbol, but your .sty file still will be fine.

\def\setdeja{\setmonofont[Scale=0.85]{DejaVu Sans Mono}}
\def\dejawarning{}

\newboolean{FontIsAvailable}
\def\testFontAvailability#1{
  \count255=\interactionmode
  \batchmode
  \let\preload=\nullfont
  \font\preload="#1" at 10pt
  \ifx\preload\nullfont \FontIsAvailablefalse
  \else \FontIsAvailabletrue \fi
  \interactionmode=\count255
}

\testFontAvailability{DejaVu Sans}
\ifFontIsAvailable
  \setdeja\else
  \def\dejawarning{\footnotetext{\bf You do not have the font DejaVu Sans Mono installed. Get it from http://dejavu-fonts.org.\\ Your .sty file created will be fine, but in the pdf there will be some symbols missing.}}\fi



\dejawarning
This is the package |alttex| which will try to give an experimental new way to write \XeLaTeX{}\footnote{If you don’t know about \XeLaTeX, see the appendix.\ref{xelatex}} code. So far it is mostly done with very dirty code and actually it’s a collection of things that come into my mind during boring lectures. Maybe someone will have fun with the following code fragments.
\tableofcontents
\newpage
\part{Introduction}
The problem I have with \LaTeX\footnote{I’ll write \LaTeX{} instead of \XeLaTeX—saves me two keystrokes. Most of the code below \emph{only} works with \XeLaTeX. If you need support for [utf8]inputenc or Lua\LaTeX, please contact the author.}  is the antique way of typing. Because most people still use a hopelessly outdated keyboard layout (»qwerty« or slightly adapted versions of that), \LaTeX{} doesn’t make use of some cool features. I’m not talking about writing chinese or arabic text! Maybe this example will make the idea clear:

In standard \LaTeX, one has to write
\begin{verbatim}
This is the normal text, then comes the itemization:
\begin{itemize}
  \item text for first item
  \item \begin{itemize}
        \item this is an item inside an item…
        \item[$\Rightarrow$] Here an item with a formula: $\int_a^b x^2 dx$
        \end{itemize}
  \item and the outer itemize goes on…
\end{itemize}
\end{verbatim}

Using this package and having a superior keyboard layout\footnote{E.\,g. the ergonomic layout Neo: \url{http://neo-layout.org/}}, you can simply write:\footnote{The lmodern font I’m using here does not have the symbol for the inner item , so we change to DejaVu Sans Mono here.}

\begin{verbatim}
This is the normal text, then comes the itemization:

• text for first item
• 

  ‣ this is an item inside an item
  ‣[⇒] Here an item with a formula: $∫_a^b x² dx$
  
• and the outer itemize goes on…

And your normal text goes on…
\end{verbatim}
Well, actually I’m lying now because this is not fully implemented so far. But it’s the aim of this package to provide this – besides many, many other funny and cool things. The aim is to offer a more „wysiwyg“ way, without loosing anything of logical markup. One still can re|\def|ine the |•| if he doesn’t like the way his items look.

\StopEventually{}
\clearpage

\part{Implementation: alttex3.sty}

\DocInput{alttex3.dtx}
\end{document}
%</driver>
%<*alttex>
% \fi
% Ok, enough blahblah, now comes the code. We begin with preamble stuff:
%    \begin{macrocode}
\ProvidesPackage{alttex3}
  [2010/01/09 v 0.c alternative way of TeXing]
\RequirePackage{expl3}
\ExplSyntaxOn
\RequirePackage{amsmath}
%    \end{macrocode}
% Checking wether \XeTeX\ or Lua\TeX\ are used.
%    \begin{macrocode}
\xetex_if_engine:F{\luatex_if_engine:F{
    \iow_term:n{^^J%
      altTeX~needs~XeLaTeX~or~LuaLaTeX~format.
    }
    \tex_errmessage:D{XeLaTeX~or~LuaLaTeX~are~required!}
    \tex_endinput:D
}}
%    \end{macrocode}
% \begin{macro}{\usepackage}
% Now, this is the first highlight. It is an extremely simple and stupid approach to load missing packages on-the-fly, just like Mik\TeX{} does. We re|\def|ine the |\usepackage| and hope, it works. Only working with texlive! If you’re using Mik\TeX{}, put a
% \begin{verbatim}  \let\usepacke\oldpackage\end{verbatim} into your preamble, \emph{directly} after loading |alttex|. If this does not work, delete the following lines from your |alttex.sty|.
%    \begin{macrocode}
\cs_set_eq:NwN\alt_oldpackage:n\usepackage
\cs_set:Npn\usepackage#1{
  \file_if_exist:nTF{#1.sty}{
    \alt_oldpackage:n{#1}
    }{
    \immediate\write18{tlmgr~install~#1}
  }
}
%    \end{macrocode}
% \end{macro}
% So far, this code seems to be a bit buggy, but it should work anyhow.
% 
% Now load some nice packages and testing wether you’re running \XeLaTeX{} or not.
%    \begin{macrocode}
\RequirePackage{exscale}
\RequirePackage{hhline}
%    \end{macrocode}
% We need |exscale| to write really big formulae, and |ifxetex| to check wether one uses the correct engine.
% \section{Textmode}
% \subsection{no escape}
% \begin{macro}{\noescape}
% You want to write plain text. Maybe you’re annoyed by always escaping characters like \_ \^ \# \& \{ \} \$ \textasciitilde{} and so on. |\noescape| allows you to never escape anything—except the \textbackslash, which still might be used for |\textit{}| or so. Or maybe not… because the \{ \} are not escaped. Have to think about this one. Maybe the \textbackslash{} will be redefined to define \{ \} by itself.
%    \begin{macrocode}
\cs_new:Npn\noescape{
  \char_set_catcode:w`\_= 11%
  \char_set_catcode:w`\^= 11%
  \char_set_catcode:w`\#= 11%
  \char_set_catcode:w`\&= 11%
  %\char_set_catcode:w`\{= 11%
  %\char_set_catcode:w`\}= 11%
  \char_set_catcode:w`\$= 11%
  \char_set_catcode:w`\~= 11%
  \char_set_catcode:w`\@= 11%
  \char_set_catcode:w`\%= 11
}
%    \end{macrocode}
% The |\makeatletter| is not necessary. But it fitted into this line, so I will leave it here.
% \end{macro}
% \begin{macro}{\oldescape}
% Of course this has to be reset when doing anything like formula, tabular etc. Maybe I will be able to change the behaviour automatically. This idea has been inspired by a discussion on the Con\TeX{}t mailinglist.
%    \begin{macrocode}
\cs_new:Npn\oldescape{
  \char_set_catcode:w`\%= 14%
  \char_set_catcode:w`\_= 8%
  \char_set_catcode:w`\^= 7%
  \char_set_catcode:w`\#= 6%
  \char_set_catcode:w`\&= 4%
  %\char_set_catcode:w`\{= 1%
  %\char_set_catcode:w`\}= 2%
  \char_set_catcode:w`\$= 3%
  \char_set_catcode:w`\~= 13%
  \makeatother%
}
%    \end{macrocode}
% \end{macro}
% \subsection{tabular}
% The way one has to type extensive tabulars is quite complex – and the resulting code is often not easy to read. I don’t have good ideas how to change this, but I’m thinking about it. Mail me any suggestions for this!
%
% This will be the first attempt to make tabulars easier: Mostly you want an |\hline| after an |\\|. So let’s try something like:\todo{I will try to implement cool stuff from the \textsf{hhline}-package.}
% \begin{macro}{\§ for \\\hhline}
% Type |\–| (an en-dash) at the end of a line, and you get an |\hhline|. Type |\=| to get a double line
%    \begin{macrocode}
\cs_set:Npn\–{\hhline}
\cs_set:Npn\={\hhline}
%    \end{macrocode}
% This is shurely not a good symbol for this purpose, but I don’t have a better idea so far. At least it’s a ”bar“, so one can guess what it should do. 
% \end{macro}
% 
% \subsection{excel tabulars}
% \begin{macro}{\exceltabular}
% Often one usese a program to calculate tabulars of numbers. To insert it into \LaTeX, one has to do some work. Here we try to copy-paste the tabular from excel, Calc or any other program to a file mytabular.txt (or any other ending). Then you say |\exceltabular{mytabular}| (you do not need the ending, therefor it doesn’t matter) and you get the tabular in a standard format. I will extend this to enable caption, variable number of columns, kind of rule used etc. This is just a very first test.
% 
% This is the definition of the command:
%    \begin{macrocode}
\cs_new:Npn\exceltabular#1{
  \char_set_catcode:w`\^^I=4\tex_relax:D
  \alt_eolintabular:%
  \begin{tabular}{|c|c|c|}\hline%
  \file_input:n{#1}%
  \end{tabular}%
  \char_set_catcode:w`\^^M=5\tex_relax:D
}
%    \end{macrocode}
% And a little helper function to make the <enter> |\active|. Again, thanks to the people on the mailinglists.
%    \begin{macrocode}
\cs_new:Npn\alt_linebreak:{\\\hline}
\tex_begingroup:D
  \tex_lccode:D`\~=`\^^M%
\tex_lowercase:D{%
  \tex_endgroup:D
  \cs_new:Npn\alt_eolintabular:{%
    \char_set_catcode:w`\^^M=\active
    \cs_set_eq:NwN~\alt_linebreak
}%
}
%    \end{macrocode}
% \end{macro}
% \subsection{tabbing}
% This will be analog to the |\exceltabular|. You write your tabbing using tabs and <enter>. That’s it :)
% \begin{macro}{\alttabbing}
% Not yet implemented!
%    \begin{macrocode}
%    \end{macrocode}
% 
% \end{macro}
% \section{Math stuff}
% \subsection{braces}
% \begin{macro}{\newbraces}
% \begin{macro}{\oldbraces}
% Now this is something most \LaTeX-beginners don’t recognize and wonder why the formula looks so ugly: The braces () do not fit to the hight of the formula. This can be achieved by putting |\left| and |\right| in front of the braces. But actually, this is annoying! In almost any case you want this behaviour, so this should be the standard. So we redefine the way braces are handled. With |\newbraces| the ( ) always fit. If you prefer the normal \LaTeX\ way, use |\oldbraces| to reset everything. This new behaviour should be extended to other characters like \verb~| [ { <~ and so on. Maybe in some later version.
% 
% I would have never been able to implement this without the help of the mailinglist members of |tex-d-l@listserv.dfn.de|!\todo{The newbraces does \emph{not} work at the moment!}
% 
% The redefinition of |\mathstrut| is necessary when using amsmath (you will use amsmath when typesetting formulae, won’t you?), because the hight of formulae is determinated by the hight of a brace. But using ( ) as |\active| characters, we need another brace here. So we take |[|. This will probably also change. But the code is working fine for ( ).\todo{Maybe one could ”temporarily hardcode“ the hight of [ and then use this…}
%    \begin{macrocode}
\def\resetMathstrut@{%
    \setbox\z@\hbox{%
      \mathchardef\@tempa\mathcode`\[\relax
      \def\@tempb##1"##2##3{\the\textfont"##3\char"}%
      \expandafter\@tempb\meaning\@tempa \relax
}%
  \ht\Mathstrutbox@\ht\z@ \dp\Mathstrutbox@\dp\z@
}

{\catcode`(\active \xdef({\left\string(}}
{\catcode`)\active \xdef){\right\string)}}

\cs_new:Npn\newbraces{
  \char_set_mathcode:w`("8000
  \char_set_mathcode:w`)"8000
}

\cs_new:Npx\oldbraces{
  \char_set_mathcode:w`(\char_value_mathcode:w`(
  \char_set_mathcode:w`)\char_value_mathcode:w`)
}
%    \end{macrocode}
% \end{macro}
% \end{macro}
% \subsection{huge display math}
% \begin{environment}{hugedisplaymath}
% Sometimes, especially in presentations, you might need an really big formula. Imagine two hours of struggle with transformations—and finally there is the beautiful formula. Now you can say\\
% \begin{center}|\begin{hugedisplaymath} E = mc^2 \end{hugedisplaymath}| \end{center}
% There should be several steps of size, maybe.
%    \begin{macrocode}
\cs_new:Npn\hugedisplaymath{
  \Huge
    \begin{equation*}
}
\cs_new:Npn\endhugedisplaymath{
  \end{equation*}
}
%    \end{macrocode}
% \end{environment}
% 
% \subsection{unicode math}
% Typing math in \TeX{} is no great fun – you have to write things like |\int| instead of |∫| and so on. Have a look at the following formula:
% 
% |\int_\infty^\infty \sum_a|
% 
% {∫}
% The code again is stolen and I don’t understand, why it does what it does, but it does it: The first argument is the character you want to use for “unicode math“, the second one is the \TeX-command.
%    \begin{macrocode}
\ExplSyntaxOff
\def\altmath#1#2{%
  \expandafter\ifx\csname cc\string#1\endcsname\relax
    \add@special{#1}%
    \expandafter
    \xdef\csname cc\string#1\endcsname{\the\catcode`#1}%
    \begingroup
      \catcode`\~\active  \lccode`\~`#1%
      \lowercase{%
      \global\expandafter\let
         \csname ac\string#1\endcsname~%
      \expandafter\gdef\expandafter~\expandafter{#2}}%
    \endgroup
    \global\catcode`#1\active
  \else
  \fi
}
\ExplSyntaxOn
%    \end{macrocode}
% We will make a switch to turn this stuff on or off, so it does not interfere with the unicode-math package. This list will increase by time. If you are missing a symbol, just send me the |\altmath{X}{\Xcode}|-line. I would be very thankful if anybody could send me a whole list of symbols!
%    \begin{macrocode}
\cs_new:Npn\makealtmath{
  \altmath{α}\alpha
  \altmath{β}\beta
  \altmath{γ}\gamma
  \altmath{δ}\delta

  \altmath{⇒}\Rightarrow
  \altmath{⇐}\Leftarrow
  \altmath{⇔}\Leftrightarrow

  \altmath{∫}\int
  \altmath{∀}\forall

  \altmath{₁}{_1}
  \altmath{₂}{_2}
  \altmath{₃}{_3}
  \altmath{₄}{_4}
  \altmath{₅}{_5}
  \altmath{₆}{_6}
  \altmath{₇}{_7}
  \altmath{₈}{_8}
  \altmath{₉}{_9}
  \altmath{₀}{_0}
}
%    \end{macrocode}
% There will be an |\makenormalmath|-switch as well.
% \subsection{Lazy underscript and superscript}
% \begin{macro}{_}
% Sometimes one has to make extensive use of subscripts and superscripts, e.\,g. when typing long formulae including tensors. Then it is a bit annoying to always write the |{}|, especially when there are only two letters in the sub/superscript. So let’s try to implement the possibility to type |$F_μν F^μν $|.
% \todo{An underscore at the end of an inline-formula has to be ended with \} or egroup. That is not nice…}
% 
% First, store the actual meaning of |_| and |^| in |\oldunderscore| and |\oldhat|. \todo{The redefinition of hat does not work because TeX uses it for definition of catcodes. There has to be a really tricky way to get around that.}
%    \begin{macrocode}
\cs_set_eq:NwN\oldunderscore_\tex_relax:D
\cs_set_eq:NwN\oldhat^\tex_relax:D
%    \end{macrocode}
% Now set |_| as |\active| char and define it the way we want it to behave. For this, we need the space char and end-of-line char to be an egroup char. So the underscript group is ended by space or eol and we don’t need to close it explicitly. (no expl3 code here, as we are dealing with the underscore. That would mess up everything …)
%    \begin{macrocode}
\ExplSyntaxOff
\catcode`\_=13
\def_{%
  \ifmmode
    \catcode`\ =2\relax%
    \catcode`\^^M=2\relax%
    \expandafter\oldunderscore\bgroup%
  \else%
    \textunderscore%
  \fi%
}

\iffalse
This does not work so far…
\catcode`\^=13
\def^{%
  \ifmmode
    \catcode`\ =2\relax%
    \catcode`\^^M=2\relax%
    \expandafter\oldhat\bgroup%
  \else%
    \oldhat%
  \fi%
}
\fi
%    \end{macrocode}
% To give the possibility to swith between normal and |alttex| behaviour, store the new underscore.
%    \begin{macrocode}
\let\advancedunderscore_
%    \end{macrocode}
% And the switches. By default, |_| is active. Type |\oldUnder| to get the normal |_|. \todo{The newUnder does not work so far.}
%    \begin{macrocode}
\def\oldUnder{
  \global\catcode`\_=8\relax
}
\def\newUnder{
  \global\let_\advancedunderscore
}
\ExplSyntaxOn
%    \end{macrocode}
% \end{macro}
% \subsection{matrices}
% This is a nice idea by Alexander Koch on \url{diskussion@neo-layout.org}. Using the unicode glyphs for writing matrices, we can make writing and readig of big matrices much easier. (In Neo, one can use the compose function to write the whole matrix by 4–5 keystrokes and then fill in the elements.) For example, say in the source:
% \begin{verbatim}
%                             ⎧a & b⎫
%   ⎛a & b⎞      ⎡a & b⎤      ⎪c & d⎪
%   ⎜c & d⎟  or  ⎢d & e⎥  or  ⎨e & f⎬
%   ⎝e & f⎠      ⎣f & g⎦      ⎪g & h⎪
%                             ⎩i & j⎭
% \end{verbatim}
% and the result will be a |bmatrix|, a |pmatrix| or a |\right\{ matrix \end{matrix}|, respectively. As \TeX\ is assumed to read from left-top to right-bottom, the matrices must not stand in a line, i.\,e. the following notation is \emph{not} possible:
% \begin{verbatim}
%     ⎛a & b⎞
% A = ⎜c & d⎟ = B
%     ⎝e & f⎠
% \end{verbatim}
% but rather you have to write
% \begin{verbatim}
% A = ⎛a & b⎞
%     ⎜c & d⎟
%     ⎝e & f⎠ = B
% \end{verbatim}
% If you have a suggestion how to enable the upper solution, please contact me, that would be an awesome thing!
% 
% One has to pay greatest attention to the different characters looking like |⎢⎥|. They are in fact \emph{different} for the three matrices! (But not in every case; I just hope the following code really works.)
%    \begin{macrocode}
\char_set_catcode:w`\⎛13
\char_set_catcode:w`\⎞13
\char_set_catcode:w`\⎟13
\char_set_catcode:w`\⎝13
\char_set_catcode:w`\⎠13
\cs_new:Npn⎛{\begin{pmatrix}}
\cs_new:Npn⎞{\\}
\cs_new:Npn⎝{}
\cs_new:Npn⎠{\end{pmatrix}}
\cs_new:Npn⎟{\\}

\catcode`\⎡13
\catcode`\⎤13
\catcode`\⎥13
\catcode`\⎣13
\catcode`\⎦13
\cs_new:Npn⎥{\\}
\cs_new:Npn⎡{\begin{bmatrix}}
\cs_new:Npn⎤{\\}
\cs_new:Npn⎣{}
\cs_new:Npn⎦{\end{bmatrix}}

\catcode`\⎧13
\catcode`\⎫13
\catcode`\⎪13
\catcode`\⎩13
\catcode`\⎭13
\catcode`\⎬13
\cs_new:Npn⎧{\left\{\begin{matrix}}
\cs_new:Npn⎫{\\\use_none:n}
\cs_new:Npn⎩{}
\cs_new:Npn⎭{\end{matrix}\right\}}
\cs_new:Npn⎪{\\\use_none:n}
\cs_new:Npn⎬{\\\use_none:n}
%    \end{macrocode}
% We need to |\use_none:n| (ignore) the next character only in this case, as the left-hand bar characters seem to be the same as the right-hand and so cause additional line breaks. This way it is robust against every strange codepoint the left-hand may have.\todo{The codepoints have to be checked very carefully! This is not what a robust solution does look like!}
% \section{Lists and such things}
% \subsection{itemize with a single character}
% \label{sec:itemize}
% \begin{macro}{• instead of \item}
% Here we use an active character (mostly a unicode character bullet •) for the whole construct. And another one for nested itemizations (like a triangular bullet \nolinebreak{\fontspec[Scale=0.8]{DejaVu Sans Mono}‣}).
% 
% This works quite fine for most \LaTeX\ classes, but \emph{not for beamer}! There, the end of |itemize| has to be given explicitly. For this, just say in the preable of your document, after loading this package: |\def\•{\end{itemize}}| and use the |\•| to end the itemization.
% 
% \begin{macro}{\newitemi}
% \begin{macro}{\newitemii}
% The following ugly peace of code is writen by me, defining the conditional insertion of the |\begin{itemize}|. This will be assigned to an active character using |\makeitemi| and |\makeitemii|, respectively.
%    \begin{macrocode}
\cs_new:Npn\outside{o}
\cs_new:Npn\inside{i}
\cs_set_eq:NwN\insideitemizei\outside
\cs_set_eq:NwN\insideitemizeii\outside
%    \end{macrocode}
% The end of itemizei and itemizeii:
%    \begin{macrocode}
\cs_new:Npn\altenditemize{
  \if\altlastitem 1%
    \let\altlastitem0%
  \else%
    \end{itemize}%
    \let\insideitemizei\outside%
  \fi%
}
%    \end{macrocode}
% Dealing with the |~| char, therefore no expl3 syntax here:
%    \begin{macrocode}
%
\ExplSyntaxOff 
\begingroup
  \lccode`\~=`\^^M%
\lowercase{%
  \endgroup
  \def\makeenteractive{%
    \catcode`\^^M=\active
    \let~\altenditemize
}%
}
\ExplSyntaxOn

\cs_new:Npn\newitemi{%
  \if_meaning:w\insideitemizei\inside%
    \cs_set_eq:NN\altlastitem1%
    \exp_after:wN\item%
  \else:%
    \begin{itemize}%
    \cs_set_eq:NN\insideitemizei\inside%
    \cs_set_eq:NN\altlastitem1%
    \makeenteractive%
    \exp_after:wN\item%
  \fi:
}

\cs_new:Npn\newitemii{
  \if_meaning:w\insideitemizeii\inside
    \exp_after:wN\item%
  \else:
    \begin{itemize}
      \cs_set_eq:NN\insideitemizeii\inside
      \exp_after:wN\item%
  \fi:
}
%    \end{macrocode}
% \end{macro}
% \end{macro}
% Ok, the following code is stolen from the |shortvrb| package, and I don’t understand anything of it. But I keep on trying… nevertheless, it’s working fine, as far as I can see.
% \begin{macro}{\makeitemi}
% \begin{macro}{\makeitemii}
% With this macro, you can define the character you want to use for first-level itemize. (Guess the sense of |\makeitemii|…) Default ist |•| for first-level and |‣| for second-level. Maybe this will be extended till fourth level. More doesn‘t seem to make any sense.
%
% Again, no expl3 due to usage of |~| here. :(
%    \begin{macrocode}
\ExplSyntaxOff
\makeatletter
\def\makeitemi#1{%
  \expandafter\ifx\csname cc\string#1\endcsname\relax
    \add@special{#1}%
    \expandafter
    \xdef\csname cc\string#1\endcsname{\the\catcode`#1}%
    \begingroup
      \catcode`\~\active  \lccode`\~`#1%
      \lowercase{%
      \global\expandafter\let
         \csname ac\string#1\endcsname~%
      \expandafter\gdef\expandafter~\expandafter{\newitemi}}%
    \endgroup
    \global\catcode`#1\active
  \else
  \fi
}

\def\makeitemii#1{%
  \expandafter\ifx\csname cc\string#1\endcsname\relax
    \add@special{#1}%
    \expandafter
    \xdef\csname cc\string#1\endcsname{\the\catcode`#1}%
    \begingroup
      \catcode`\~\active  \lccode`\~`#1%
      \lowercase{%
      \global\expandafter\let
         \csname ac\string#1\endcsname~%
      \expandafter\gdef\expandafter~\expandafter{\newitemii}}%
    \endgroup
    \global\catcode`#1\active
  \else
  \fi
}
\ExplSyntaxOn
%    \end{macrocode}
% \end{macro}\end{macro}
% Now there are the two helperfunctions – no guess what they are really doing.
%    \begin{macrocode}
\cs_new:Npn\add@special#1{%
  \rem@special{#1}%
  \exp_after:wN\cs_gset_nopar:Npn\exp_after:wN\dospecials\exp_after:wN
{\dospecials \do #1}%
  \exp_after:wN\cs_gset_nopar:Npn\exp_after:wN\@sanitize\exp_after:wN
{\@sanitize \@makeother #1}}
\cs_new:Npn\rem@special#1{%
  \cs_set:Npn\do##1{%
    \if_num:w`#1=`##1 \else: \exp_not:N\do\exp_not:N##1\fi:}%
  \cs_set_nopar:Npx\dospecials{\dospecials}%
  \tex_begingroup:D
    \cs_set:Npn\@makeother##1{%
      \if_num:w`#1=`##1 \else: \exp_not:N\@makeother\exp_not:N##1\fi:}%
    \cs_set_nopar:Npx\@sanitize{\@sanitize}%
  \tex_endgroup:D}
%    \end{macrocode}
% \end{macro}
% \subsection{enumerate with a single character}
% \begin{macro}{¹, ²}
% And we do just the same stuff with |\enumerate|. But here we take the character |¹| as first level item, the |²|\footnote{Maybe this is a very stupid idea, because now the |²| cannot be used as a square in mathmode. Of course there could be a test |ifmmode|, but I rather would like to find a better character for |enumerate|.} as second level etc. This may be confusing some way, but just try it.
%
% For the implementation: copy-pasted the code above, nothing interesting so far.
%    \begin{macrocode}
\cs_new:Npn\¹{\end{enumerate}}
\cs_new:Npn\²{\end{enumerate}}

\cs_set_eq:NN\insideenumi\outside
\cs_set_eq:NN\insideenumii\outside

\cs_new:Npn\newenumi{
  \if_meaning:w\insideenumi\inside
    \exp_after:wN\item%
  \else:
    \begin{enumerate}
      \cs_set_eq:NN\insideenumi\inside
      \exp_after:wN\item%
  \fi:
}

\cs_new:Npn\newenumii{
  \if_meaning:w\insideenumii\inside
    \exp_after:wN\item%
  \else:
    \begin{enumerate}
      \cs_set_eq:NN\insideenumii\inside
      \exp_after:wN\item%
  \fi:
}
%    \end{macrocode}
% We use the same methods as above, still not understanding, what they are doing. Just changing two lines of code and hoping, everything will be fine. Again, no expl3.
%    \begin{macrocode}
\ExplSyntaxOff
\makeatletter
\def\makeenumi#1{%
  \expandafter\ifx\csname cc\string#1\endcsname\relax
    \add@special{#1}%
    \expandafter
    \xdef\csname cc\string#1\endcsname{\the\catcode`#1}%
    \begingroup
      \catcode`\~\active  \lccode`\~`#1%
      \lowercase{%
      \global\expandafter\let
         \csname ac\string#1\endcsname~%
      \expandafter\gdef\expandafter~\expandafter{\newenumi}}%
    \endgroup
    \global\catcode`#1\active
  \else
  \fi
}

\def\makeenumii#1{%
  \expandafter\ifx\csname cc\string#1\endcsname\relax
    \add@special{#1}%
    \expandafter
    \xdef\csname cc\string#1\endcsname{\the\catcode`#1}%
    \begingroup
      \catcode`\~\active  \lccode`\~`#1%
      \lowercase{%
      \global\expandafter\let
         \csname ac\string#1\endcsname~%
      \expandafter\gdef\expandafter~\expandafter{\newenumii}}%
    \endgroup
    \global\catcode`#1\active
  \else
  \fi
}
\makeatother
\ExplSyntaxOn
%    \end{macrocode}
% \end{macro}
% Finally, we set the default characters for the items and enumerations:
%    \begin{macrocode}
\makeitemi•
\makeitemii‣
\makeenumi¹
\makeenumii²
%    \end{macrocode}
% And that’s it.\begin{center}\Large Happy alt\TeX{}ing!\end{center}
% \section{Known Bugs}
% This should be a list of serious bugs. Please report any of them to me!
% \begin{itemize}
% \item Itemize does not work correctly in beamer. Use |\•| at the end of your itemize. (see section \ref{sec:itemize})
% \end{itemize}
% \newpage\appendix
% \section*{A very short introduction to \XeLaTeX}
% \label{xelatex}
  % Everything you have to know about \XeLaTeX\ to use this package: Write your \LaTeX\ file just as you are used to. But save it as utf8-encoded, and say
% \begin{verbatim}\usepackage{xltxtra}\end{verbatim}
% instead of 
% \begin{verbatim}\usepackage[latin1]{inputenc} and \usepackage[T1]{fontenc}\end{verbatim}
% 
% This loads some files that provide all the cool stuff \XeLaTeX\ offers. You don’t have to take care of letters \TeX\ would not understand – \XeTeX\ understands every character you type. But sometimes the font may not have the symbol for this – then you can use |\fontspec{fontname}|, where |fontname| is the name of a font on your system, e.\,g. |Arno Pro|, |Linux Libertine|, |LT Zapfino One| etc.
% 
% Then, you compile your document with the command |xelatex file.tex|, instead of |latex file.tex| and you get a pdf as output. Mostly, your editor will not have a shortcut to start \XeLaTeX. In that case, you have to compile via the command line. If you know your editor well enough, you may be able to create a shortcut that will run |xelatex file.tex| for you. Notice that you will need an editor that is utf8-capable! One last warning: While \XeTeX\ is not an pdf\TeX\ successor, you cannot use microtypographic extensions. Maybe in the future there will be an implementation that uses advanced OpenType-features, but at the moment there is no microtypography possible!
% 
% If you have any trouble using \XeLaTeX, just e-mail me!
% \newpage
% \section*{todo}
% Here a section with some ideas that could be implemented.
% \begin{itemize}
% \item Use |²| as square in mathmode and possibly |¹| as |\footnote|?
% \item Do something to enable easy tabular
% \item If there is only one char after an |_|, there should no space be needed.
% \item Maybe there could be a ConTeXt-version of this file.
% \end{itemize}
% \iffalse
%</alttex>
% \fi
% \iffalse
%<*sample>
\documentclass{article}
\usepackage[english]{babel}
\usepackage{geometry}
\usepackage{
alttex,
blindtext,
dtklogos,
fancyhdr,
multicol,
lmodern,
shortvrb,
xltxtra
}

\MakeShortVerb|

\newenvironment{sample}[1]{
  \subsection*{#1}
  \hrule
  \label{#1}
  \begin{minipage}{\textwidth}
  \begin{multicols}{3}}
{\end{multicols}
  \end{minipage}\vspace{2cm}}

\fancyhead{\normalfont\qquad alttex\hfill normal \hfill result\qquad}
\pagestyle{fancy}

\setmonofont[Scale=0.8]{DejaVu Sans} % the standard fonts don’t have the nice symbols!

\title{A short sample document for the |alttex| package}
\author{Arno L. Trautmann\thanks{arno.trautmann@gmx.de}}
\date{\today}

\begin{document}
\maketitle
This document shows applications of the possibilities the |alttex| packages offers for writing code in \XeLaTeX.

There are three columns: The left one shows the input code using |alttex|, the middle one shows the corresponding ”normal“ input, and on the right-hand side there is the output of that code. The output is written in the code on the left-hand side, so I am not cheating here…\vspace{2cm}

\sample{noescape}
\noescape
\begin{verbatim}
This is a short text including
some characters like an @,
a §, 100% unescaped, even {}
and ~ or a &
without a backslash.
\end{verbaatim}
\vfill
\oldescape
\begin{verbatim}
This is a short text including
some characters like an @,
a §, 100\% unescaped, even \{\}
and \textasciitilde\ or a
\& without a backslash.
\end{verbatim}
\vfill
\noescape
This is a short text including
some characters like an @,
a §, 100% unescaped, even {}
and ~ or a &
without a backslash.
\oldescape
\endsample
\makealtmath

\sample{general math}
|\[∫_-∞ ^∞ ∀\]|

\ \vfill
|\[\int_{-\infty}^{\infty}\]|
\ \vfill
\[∫_-∞ \ \]% ^∞ ∀\]
\endsample

\sample{itemize}
\begin{verbatim}
A short sample text

• first item
• second item

  ‣ deeper item
  ‣ second deeper item
  \end{itemize}
\end{itemize}
\end{verbatim}
\ \vfill
\begin{verbatim}
A short sample text
\begin{itemize}
\item first item
\item second item
\begin{itemize}
  \item deeper item
  \item second deeper item
  \end{itemize}
\end{itemize}
\end{verbatim}
\ \vfill
A short sample text
\def\‣{\end{itemize}}

• first item
• second item

  ‣ deeper item
  ‣ second deeper item
  \‣

\endsample

\sample{enumerate}
\begin{verbatim}
A short sample text

¹ first item
¹ second item

  ² deeper item
  ² second deeper item
  \end{enumerate}
\end{enumerate}
\end{verbatim}
\vfill
\begin{verbatim}
A short sample text
\begin{enumerate}
\item first item
\item second item
\begin{enumerate}
  \item deeper item
  \item second deeper item
  \end{enumerate}
\end{enumerate}
\end{verbatim}
\vfill
A short sample text

¹ first item
¹ second item

  ² deeper item
  ² second deeper item
  \end{enumerate}
\end{enumerate}
\endsample

\sample{huge display math}
\begin{verbatim}
\begin{hugedisplaymath}
E = mc^2
\end{hugedisplaymath}
\[E = mc^2\]
\end{verbatim}
\vfill
\begin{verbatim}
\Huge
\[ E = mc^2\]
\normalsize
\[E = mc^2\]
\end{verbatim}
\vfill
\begin{hugedisplaymath}
E = mc^2
\end{hugedisplaymath}
\[E = mc^2\]
\endsample
\end{document}
%</sample>
% \fi
% \Finale
\endinput